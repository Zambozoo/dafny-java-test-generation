\documentclass[11pt,onecolumn,notitlepage]{article}

\usepackage[margin=1in]{geometry}
\usepackage{appendix}
\usepackage[numbib]{tocbibind}
\usepackage{paralist}
\usepackage{listings}
\usepackage{graphics}
\usepackage[usenames,dvipsnames,svgnames,table]{xcolor}
\usepackage{comment}
\usepackage{url}

\definecolor{dkgreen}{rgb}{0,0.6,0}
\definecolor{mauve}{rgb}{0.58,0,0.82}
\definecolor{light-gray}{gray}{0.88}

\lstset{frame=none,
  language=Java,
  aboveskip=3mm,
  belowskip=3mm,
  showstringspaces=false,
  columns=flexible,
  numbers=none,
  numberstyle=\tiny\color{gray},
  commentstyle=\color{dkgreen},
  stringstyle=\color{mauve},
  breaklines=true,
  breakatwhitespace=true,
  tabsize=2,
  morekeywords={function, method, then, reads, modifies, requires, ensures, var, assume, assert}
}

\newcommand{\figref}[1]{Fig.~\ref{#1}}

\makeatletter
\renewcommand{\maketitle}{\bgroup\setlength{\parindent}{0pt}
\begin{flushleft}
  \Large{\textbf{\@title}}
  
  \normalsize{\@author}
\end{flushleft}\egroup
}
\makeatother

% \pagestyle{empty}
\begin{document}

\title{Symbolic Execution for Generating Java Tests from Dafny Models}
\author{\hfill \\
        PI: Eric Mercer, Associate Professor, Computer Science, Brigham Young University\\
        Cash funding needed: \$53,420 \\
        AWS Promotional Credits needed: none\\
        Amazon Contact: Sean McLaughlin, Aleks Charkarov, Matthias Schalaipfer, \{seanmcl, aleksach, schlaipf\}@amazon.com}

\maketitle

\begin{abstract}
  Dafny is a modeling and specification language with a static verifier to prove that a program implements its specification. Although it includes a compiler to \texttt{Java}, the resulting programs lack efficiency, readability, and any guarantee of preserving correctness. AWS is working on a specialized compiler so that Dafny can be used as a modeling language for web-services. The work in this proposal is to generate a \texttt{JUnit} test suite from a Dafny model that gives object branch coverage (e.g., includes short-circuit evaluation) and specification coverage (e.g.,, negative and positive tests for post-conditions). Such a test suite provides some assurance that the compiler preserves the static correctness properties of the Dafny source model.
\end{abstract}

\providecommand{\keywords}[1]{\noindent\textbf{Keywords -- } \textit{#1}}
\keywords{Contract programming, automatic test generation, symbolic execution}

\section*{Introduction}
\emph{Contract programming} prescribes that designers write a precise interface specification for software components that can be used to reason about implementations. A \emph{contract} is a declarative specification of what the software component computes and under what conditions that computation is correct. It defines required \emph{pre-conditions} for a component's state and input at invocation, and it defines the guaranteed \emph{post-conditions} of the component's state and outputs after invocation. Contract programming reduces defects and increases reliability by giving pause to the designer to think critically about, and state formally, the intent of each software component in context of the system and by providing early in the design process a robust foundation for automated reasoning. 

\emph{Dafny} is an expressive programming language for the static verification of programs. It is a sequential imperative object-oriented language with support for generics, inductive data-types, co-inductive data-types, and most importantly, constructs for formal specification. These constructs include contracts, \emph{frames} to define read and write sets, loop-invariants, and ranking functions in the form of \emph{decreases} clauses for termination. The Dafny verifier implements a deductive \emph{weakest pre-condition} calculus to statically prove a Dafny program terminates and adheres to its specification.  

The Dafny compiler targets a handful of backend languages, including \texttt{Java}, to implement a verified Dafny program. The compiler outputs source level code which can then be compiled to the intended framework. There is no guarantee that the meaning of the original Dafny program is preserved through compilation, and in general, the compiler does not produce readable, efficient, or idiomatic code. Such a final program is not suitable for deployment in any high assurance system. AWS is working on a specialized compiler from Dafny to \texttt{Java} for such systems so that Dafny can be used as a modeling language for web-services.  The subject of this proposal is how to provide some further affirmation that the resulting compiled \texttt{Java} program still conforms to its original Dafny specification. 

Proving that compilation exactly preserves the meaning of the original Dafny program, while an interesting and hard problem, is unlikely to be feasible in the near term (see CakeML or CompCert as examples \cite{cakeml, compcert}). An acceptable alternative is to test to some predefined standard suitable for high assurance systems. This proposal is to automatically generate tests from a Dafny program sufficient to give \emph{object branch coverage} and \emph{specification coverage}. Object branch coverage is a common \emph{white-box} criteria that means that at the compiled object code level, every branch outcome is exercised by at least one test to account for short-circuiting of Boolean operations (note that it is possible to further strengthen the test suite to meet \emph{Modified Condition/Decision Coverage} if desired \cite{7194601}). It is stronger than \emph{statement coverage} and \emph{branch coverage}, so a test suite that meets object branch coverage also meets those standards. Specification coverage is a \emph{black-box} criteria based on \emph{input partitioning} where there is a negative test for each pre-condition and a positive test for each post-condition in the specification (note that it may be possible to further strengthen this to \emph{boundary value analysis} if desired). These two coverage criteria provide some assurance that the compiled Dafny program in \texttt{Java} still implements its specification.

It's worth nothing that the value of Dafny's static verification is only as good the specification. It is easy to write weak contracts that say little about intent and are trivial to prove. It is also not hard to write strong contracts that are as complex as the actual code and hard to prove. Experience with engineers in Collins Aerospace and with undergraduates learning Dafny shows that it is not uncommon to leave out critical details that should be in a contract, believe a contract is saying something that it is not, or simply not realize a contract is vacuously true by virtue of how implications are used. Testing and debugging contracts is an interesting and hard problem. 

A step toward testing contracts is to convert manually written \texttt{Java} tests to Dafny tests. Suppose there is an existing \texttt{Java} implementation of some service that is used as the basis for a Dafny model. The designer writes the specification for the Dafny model to statically prove each software component based on its \texttt{Java} counterpart. How does this designer know if the specification is strong enough to be useful? One answer is that the specification should be at least strong enough to discern negative and positive tests from the original \texttt{Java} implementation. If it is too weak to discern such tests, then it can be strengthened as needed. In this way, a designer is able to leverage the original \texttt{Java} tests to write the specification for the Dafny model (see \cite{repo} for example).
 
\subsection*{Related Work}

Many different tools and languages integrate contract programming in different ways (e.g., Python, Typescript, Java, C/C++, Rust, Eiffel, Go, AGREE, etc.). Contracts can be higher-order or first-order depending on the language. Contracts are often used as programming annotations to improve readability of the code and enhance maintainability (e.g. Racket \cite{10.1145/3022670.2951930,10.1145/583852.581484,10.1145/2034574.2034800}). This role is especially valuable in untyped script languages where making sense of expected types can be challenging if not the original author (e.g. Python, Javascript, Typescript, Perl). 

Contracts form the basis of runtime monitors to check pre-conditions and assert post-conditions \cite{10.1007/978-3-642-28869-2_11}. Monitors can be removed by proving an implementation adheres to its contract. Such proofs can be accomplished with static program analysis (abstract interpretation or type systems) \cite{10.1145/3158139}, model checking (k-induction in Lustre/AGREE) \cite{10.1007/978-3-642-23702-7_26,10.1007/978-3-319-96142-2_3, 10.1007/978-3-642-28891-3_13, 10.1007/3-540-48249-0_34, 10.1007/978-3-319-41540-6_29}, or deductive reasoning (weakest pre-condition or strongest post-condition calculus) \cite{10.1007/3-540-45314-8_21, Huisman2016, 10.1007/978-3-642-27705-4_7, DBLP:series/lncs/10001, 10.1007/978-3-030-03421-4_4}. Counter-examples from failed proofs often are some of the very best tests (e.g., generating tests from counter-examples) \cite{Billeter:Thesis:2008}.

Contracts are equally useful for automated testing \cite{10.1007/978-3-540-69507-3_9, Ciupa05automatictesting, 8471992, 8972014}. A post-condition is a pre-defined, and convenient, test oracle. The pre-condition is an equally convenient, and useful, input constraint for property based testing and fuzz testing. The contracts can even be employed in mutation analysis to generate tests to show the refinement relation between the contract and the implementing code \cite{KRENN200971}. 

There are few examples of fully verified compilers. A notable example is perhaps in the SeL4 operating system and the CakeML framework \cite{10.1145/1631687.1596566,ESOP18}. There is also the CompCert C compiler that is a high-assurance compiler for embedded systems \cite{Leroy-backend,2008-Leroy-Blazy-memory-model}. Neither of these are suitable for a Java deployment.

\section*{Methods}

\newsavebox{\boxT}
\begin{lrbox}{\boxT}
\begin{lstlisting}
class T {
  var f : int; var l : int;  var v : bool;

  function method isF(f : int) : bool
  reads `f, `v { 
    (v && (f == this.f))
  }

  function method getL() : int
  reads `l { 
    l
  }

  method setV(f : int)
  ensures v == old(isF(f))
  modifies `v {
    v := isF(f);
  }
}
\end{lstlisting}
\end{lrbox}

\newsavebox{\boxI}
\begin{lrbox}{\boxI}
\begin{lstlisting}
class I {
  var a : bool; var s : bool; var l : int;

  method o(t : T, f : int)
  requires !a && !s
  ensures s == (old(t.isF(f)) && l >= t.getL())
  ensures !(old(t.isF(f)) && l >= t.getL()) 
          ==> 
          (a == t.isF(f))
  modifies t, `a, `s {
    if (t.isF(f) && l >= t.getL()) {
      s := true;
    } else {
      s := false;
      t.setV(f);
      a := t.isF(f);
    }
  }
}
\end{lstlisting}
\end{lrbox}

\newsavebox{\boxa}
\begin{lrbox}{\boxa}
\begin{lstlisting}
assume (t.isF(f));
assume (l >= t.getL());
s := true;
\end{lstlisting}
\end{lrbox}

\newsavebox{\boxb}
\begin{lrbox}{\boxb}
\begin{lstlisting}
assume (t.isF(f));
assume !(l >= t.getL());
s := false;
t.setV(f);
a := t.isF(f);
\end{lstlisting}
\end{lrbox}

\newsavebox{\boxc}
\begin{lrbox}{\boxc}
\begin{lstlisting}
assume(!t.isF(f));
s := false;
t.setV(f);
a := t.isF(f);
\end{lstlisting}
\end{lrbox}

Automatic test generation in this work is accomplished with either the Dafny verifier using its weakest pre-condition calculus or symbolic execution in Symbolic Java Pathfinder \cite {10.1007/978-3-540-71209-1_12}. Both approaches rely on constraint solving with a backend \emph{Satisfiability Modulus Theories} (SMT) solver. This presentation uses \emph{symbolic execution} but it is not clear that one approach is more advantageous than another. Indeed, there is some obvious advantage with integrating test generation into the Dafny Verifier as it already provides much of the needed functionality.

\begin{figure}
  \begin{center}
    \setlength{\tabcolsep}{20pt}
    \begin{tabular}{cc}
      \scalebox{0.85}{\usebox{\boxT}} & \scalebox{0.85}{\usebox{\boxI}}
    \end{tabular}
  \end{center}
\caption{An illustrative Dafny model.}
\label{fig:dafny}
\end{figure}

\begin{figure}
  \begin{center}
    \setlength{\tabcolsep}{20pt}
    \begin{tabular}[t]{ccc}
      \scalebox{0.85}{\usebox{\boxa}} & \scalebox{0.85}{\usebox{\boxb}} & \scalebox{0.85}{\usebox{\boxc}} \\
      (a) & (b) & (c)
    \end{tabular}
  \end{center}
\caption{Paths to test for object branch coverage. (a) The true path. (b) The false path on the second condition. (c) The false path on the first condition.}
\label{fig:paths}
\end{figure}

Consider the Dafny program in \figref{fig:dafny}(a). The program consists of a token class, \texttt{T}, and an ID station class, \texttt{I}. The token stores a fingerprint (\texttt{f}), an access clearance level (\texttt{l}), and a bit indicating if the token is valid (\texttt{v}). \emph{Function-methods} are read only with the frame making clear what is read. These check a match between a passed in fingerprint to the stored fingerprint (\texttt{T.isF(f:int)}) and return the clearance level (\texttt{T.getL()}). The last method (\texttt{T.setV(f:int)}) side-effects as indicated by the modifies clause. It invalidates the token on a mismatched finger print.

The ID station class stores the alarm state (\texttt{a}), the state of a door, (\texttt{s}), and the minimum required clearance level needed to access the door (\texttt{l}). The one method, \texttt{o(t : T, f : int)}, in the class arbitrates access to the door by checking a scanned fingerprint against the stored fingerprint in the token. The specification declares the behavior of the door and alarm. The pre-conditions (\texttt{requires}) require the alarm to be off and the door closed. The post-conditions (\texttt{ensures}) merit some discussion. The first ensures says the door only opens when the fingerprint matches the token and the token is within the clearance level. The second ensures is more subtle but intuitively the alarm sounds if the fingerprint does not match the token. Dafny is able to prove the code implements the specification with deductive reasoning.

The goal is to generate tests for object branch coverage of \texttt{I.o(t : T, f : int)} with symbolic execution. Symbolic execution enumerates all the control flow paths in the method, and for each path, it generates a \emph{path constraint} over the object state and primary inputs that must be satisfied to activate the path. Each path constraint is dispatched to a backend SMT solver to find the initial state values and primary inputs, if any, needed to activate the path. Paths are enumerated until there is a sufficient set to satisfy the coverage criteria.

There are three paths needed for object branch coverage for \texttt{I.o(t : T, f : int)}. For simplicity, these are shown as \emph{straight line programs} in \figref{fig:paths}. A straight line program has no branching and uses the \emph{assume} keyword to assert constraints that must hold for program execution to continue. 

Dafny specifications provide important information for simplifying symbolic execution:
\begin{compactitem}
  \item Pre and post conditions are summaries that can be used at call sites.
  \item Frames convey dependencies when call-sites need to be expanded with summaries and why.
  \item Loop invariants and ranking functions in decrease clauses guarantee termination.
\end{compactitem}
Such detailed specifications address the major hurdles to symbolic execution.

The straight line program in \figref{fig:paths}(a) is illustrative. Symbolic execution initializes the program state and input with \emph{uninterpreted functions} (i.e., unconstrained variables that are able to assume any value) and executes the program to track the state and variable definition changes in terms of the uninterpreted functions. At the same time, it propagates the path constraint to capture the conditions that must be met by the state and variables to continue along the straight line program.

Symbolic execution begins by initializing a symbol table for variables in the current frame. This table is used to create the initial path constraint and to track changes to variables in the program path. The initial path constraint for \figref{fig:paths}(a) is $\neg a_0\ \wedge\ \neg s_0$ to encapsulate the pre-condition in the specification where $a_0$ and $s_0$ are symbolic versions of the fields \texttt{a} and \texttt{s} in the symbol table. Novel in this proposal is the use of the Dafny frame in the specification for dependency analysis to determine that \texttt{t.isF(f)} and \texttt{t.getL()} are independent and not side-effecting; as such, these can be represented with uninterpreted functions, $\mathit{t.isF}_0$ and $\mathit{t.getL}_0$, in the symbolic execution. The two methods, with the represented symbolic variables, are added to the symbol table.

Symbolic execution proceeds by building up the path constraint as it encounters the assume statements. The new path constraint after the assume statements needs to express the object state that must exist before the assignment to field \texttt{s}: $\neg s_0\ \wedge\ \neg a_0\ \wedge \mathit{t.isF}_0 \wedge (l\ \ge\ \mathit{t.getL}_0)$. Symbolic execution then applies to the symbol table the assignment to \texttt{s} changing its entry to be the value \texttt{true}. That value would be used for \texttt{s} if the program continued.

The path constraint is passed to the backend SMT solver to generate inputs for the test: $a_0 = \mathrm{false}$, $s_0 = \mathrm{false}$, $\mathit{l} = 0$, $\mathit{t.isF}_0 = \mathrm{false}$, and $\mathit{t.getL}_0 = 0$. The generated JUnit test uses a \emph{mock} object for an instance of \texttt{T} for the test and directs the mock to return true whenever \texttt{T.isF(f)} is invoked and zero for whenever \texttt{T.getL()} is invoked.\footnote{Mockito is to be used for mocks.} 

The post-conditions define the oracle for the test. What is interesting in this test is that the ensures statement in the specification makes reference to the \emph{old} value of \texttt{t.isF(f)}. The old value is the value before the method call. The test must capture the value in a local variable before the call to the method under test in order to use that old value in the assertion. Dafny disallows any side-effecting calls in the specification so it is never the case that capturing the old value changes the object state. \figref{fig:test}(a) is the final test expressed as a Dafny program. The assume statements fill the role of mocks. Although a simple example, it illustrates the utility of the frame in the Dafny specification to simplify the path constraint by abstracting out needless details through dependency analysis. This test program can then be used to generate a \emph{JUnit} test for the \texttt{Java} implementation.

\newsavebox{\boxd}
\begin{lrbox}{\boxd}
\begin{lstlisting}
method test_path0() {
  var i: I := new I; 
  i.a := false; i.s := false; i.l := 0;
  var t: T := new T; 
  var f: int;
  assume (t.isF(f)); assume (t.getL() == 0);
  var old_isF := t.isF(f);
  
  i.o(t, f);
  
  assert(i.s == (old_isF && i.l >= t.getL()));
  assert(!(old_isF && i.l >= t.getL()) 
           ==> (i.a == t.isF(f)));
}
\end{lstlisting}
\end{lrbox}


\newsavebox{\boxe}
\begin{lrbox}{\boxe}
\begin{lstlisting}
method test_path1() {
  var i: I := new I; 
  i.a := false; i.s := false; i.l := 0;
  var t: T := new T; t.v := true; t.f := 0;
  var f: int := 0;
  assert (t.isF(f)); assume (t.getL() == 0);
  var old_isF := t.isF(f);

  i.o(t, f);
  
  assert(i.s == (old_isF && i.l >= t.getL()));
  assert(!(old_isF && i.l >= t.getL()) 
           ==> (i.a == t.isF(f)));
}
\end{lstlisting}
\end{lrbox}

\begin{figure}
  \begin{center}
    \setlength{\tabcolsep}{10pt}
    \begin{tabular}{cc}
      \scalebox{0.85}{\usebox{\boxd}} & \scalebox{0.85}{\usebox{\boxe}} \\
      (a) & (b)
    \end{tabular}
  \end{center}
\caption{Generated tests. (a) The true path. (b) A false path.}
\label{fig:test}
\end{figure}

Consider the slightly more complex path in \figref{fig:paths}(b). Here the dependency analysis identifies the relation between \texttt{t.setV(f)} and \texttt{t.isF(f)} with the former modifying a field that the latter reads. As before, \texttt{t.getL()} is independent. This proposal takes inspiration from SMT solvers by first solving the constraint problem with uninterpreted functions, and then, with dependency analysis, solve any secondary constraint problems. As such, the path constraint at the end of the method is $\neg s_0\ \wedge\ \neg a_0\ \wedge \mathit{t.isF}_0 \wedge \neg(l\ \ge\ \mathit{t.getL}_0)$ with a satisfying assignment of $a_0 = \mathrm{false}$, $s_0 = \mathrm{false}$, $\mathit{l} = 0$, $\mathit{t.isF}_0 = \mathrm{false}$, and $\mathit{t.getL}_0 = 1$. The secondary constraint is solved using the specification for \texttt{t.setV(f)}: $t.v_0 \wedge (f_0 == t.f_0)$. These assignments are $t.v_0 = 0$, $t.f_0 = 0$, and $f_0 = 0$ for the final test in \figref{fig:test}(b). The test asserts rather than assumes the call to \texttt{t.isF(f)} has the correct value.

\begin{comment}
Symbolic execution must track side-effecting calls and introduce new uninterpreted functions as appropriate. For example, a second call to \texttt{T.isF(f : int)} should create a new uninterpreted function since its value might be different from the first call, and such information should be captured in the symbol table and path constraint as appropriate. Also note that it is possible to skip the secondary constraint problem with the flattened contracts to rather over-approximate actual program behavior using mocks. The mocks can be configured to return the needed sequence of program values to test the path. Depending on the code being tested, such an over-approximation may be sufficient for the test though there are two things to consider: first, what if such a path is not feasible in any real execution (i.e., there does not exist any input that would exercise that path); and second, what does a passing or failing test mean? Here it is not clear if such an over-approximation is sound and the question merits further investigation.
\end{comment}

Not discussed thus far are quantifiers in the specification. These may require more advanced theories in the SMT solver if they appear in the path constraint because of a pre-condition, or they can be flattened. Quantifiers in the post-condition are flattened or translated into appropriate looping structures. Also, not discussed yet are looping structures, inductive types, and recursion. Loops in Dafny include loop invariants and decreases clauses for termination. These specifications should be sufficient for symbolic execution.  Inductive types, recursion, and other advanced Dafny features (lemmas, ghosting, etc.) are not considered in this proposal. 

\section*{Expected Results}
A tool that generates a \texttt{JUnit} test-suite for object branch and specification coverage from a verified Dafny program is the expected outcome with appropriate publications. The first quarter is to generate tests from straight line programs tackling the dependency analysis and symbolic execution. The second quarter addresses loops, and the use of loop-invariants, in the test generation. The third quarter implements path enumeration. The fourth quarter targets case studies and publications.

\section*{Funds Needed}

The \$53,420 in requested funds provide direct support for two graduate students. The amount is based on both the cost of wages and tuition. Base wage is \$20.00 per hour for 20 hours per week for 16 weeks for the fall and winter semesters, and it allocates 40 hours per week for the spring/summer semester for 16 weeks. That is 1,920 total hours for a total wage cost of \$38,400. Full-time graduate student tuition for 2020-2021 is \$7,510 per student for a total tuition cost of \$15,020. 

\appendix
\appendixpage

\bibliographystyle{plain}
\bibliography{proposal}

\section{CV of PI}

\section{Previously Funded Project Summary}
Nothing to report.
\end{document}